%%%%%%%%%%%%%%%%%%%%%%%%%%%%%%%%%%%%%%%%%%%%%%%%%%%%%%%%%%%%%%%%%%%%
%% I, the copyright holder of this work, release this work into the
%% public domain. This applies worldwide. In some countries this may
%% not be legally possible; if so: I grant anyone the right to use
%% this work for any purpose, without any conditions, unless such
%% conditions are required by law.
%%%%%%%%%%%%%%%%%%%%%%%%%%%%%%%%%%%%%%%%%%%%%%%%%%%%%%%%%%%%%%%%%%%%

\documentclass[
  digital, %% This option enables the default options for the
           %% digital version of a document. Replace with `printed`
           %% to enable the default options for the printed version
           %% of a document.
  twoside, %% This option enables double-sided typesetting. Use at
           %% least 120 g/m² paper to prevent show-through. Replace
           %% with `oneside` to use one-sided typesetting; use only
           %% if you don’t have access to a double-sided printer,
           %% or if one-sided typesetting is a formal requirement
           %% at your faculty.
  table,   %% This option causes the coloring of tables. Replace
           %% with `notable` to restore plain LaTeX tables.
  nolof,     %% This option prints the List of Figures. Replace with
           %% `nolof` to hide the List of Figures.
  nolot,     %% This option prints the List of Tables. Replace with
           %% `nolot` to hide the List of Tables.
  %% More options are listed in the user guide at
  %% <http://mirrors.ctan.org/macros/latex/contrib/fithesis/guide/mu/fi.pdf>.
]{fithesis3}
%% The following section sets up the locales used in the thesis.
\usepackage[resetfonts]{cmap} %% We need to load the T2A font encoding
\usepackage[T1,T2A]{fontenc}  %% to use the Cyrillic fonts with Russian texts.
\usepackage[
  main=english, %% By using `czech` or `slovak` as the main locale
                %% instead of `english`, you can typeset the thesis
                %% in either Czech or Slovak, respectively.
  english, german, russian, czech, slovak %% The additional keys allow
]{babel}        %% foreign texts to be typeset as follows:
%%
%%   \begin{otherlanguage}{german}  ... \end{otherlanguage}
%%   \begin{otherlanguage}{russian} ... \end{otherlanguage}
%%   \begin{otherlanguage}{czech}   ... \end{otherlanguage}
%%   \begin{otherlanguage}{slovak}  ... \end{otherlanguage}
%%
%% For non-Latin scripts, it may be necessary to load additional
%% fonts:
\usepackage{paratype}
\def\textrussian#1{{\usefont{T2A}{PTSerif-TLF}{m}{rm}#1}}
%%
%% The following section sets up the metadata of the thesis.
\thesissetup{
    date          = \the\year/\the\month/\the\day,
    university    = mu,
    faculty       = fi,
    type          = mgr,
    author        = Bc. Andrej Staruch,
    gender        = m,
    advisor       = {RNDr. Marek Kumpošt, Ph.D.},
    title         = {Phishing-detection application},
    TeXtitle      = {Phishing-detection application},
    keywords      = {phishing, anti-spam, application},
    TeXkeywords   = {phishing, anti-spam, application},
    abstract      = {The goal of this master's thesis is to develop and implement an application, which will recognize potential risk for a given URL from web traffic based on extendable series of individual tests. The result of weighted tests is 'phishing score' and one of three actions for an URL: bypass, warn and block. The source for  tests should be this following work: https://is.muni.cz/auth/th/dyi2v/Diplomova\_prace.pdf
    
    },
    thanks        = {These are the acknowledgements for my thesis, Mgr. Karol Kubanda, učo 143339 (konzultant) },
    bib           = example.bib,
}
\usepackage{makeidx}      %% The `makeidx` package contains
\makeindex                %% helper commands for index typesetting.
%% These additional packages are used within the document:
\usepackage{paralist} %% Compact list environments
\usepackage{amsmath}  %% Mathematics
\usepackage{amsthm}
\usepackage{amsfonts}
\usepackage{url}      %% Hyperlinks
\usepackage{markdown} %% Lightweight markup
\usepackage{listings} %% Source code highlighting
\lstset{
  basicstyle      = \ttfamily,%
  identifierstyle = \color{black},%
  keywordstyle    = \color{blue},%
  keywordstyle    = {[2]\color{cyan}},%
  keywordstyle    = {[3]\color{olive}},%
  stringstyle     = \color{teal},%
  commentstyle    = \itshape\color{magenta}}
\usepackage{floatrow} %% Putting captions above tables
\floatsetup[table]{capposition=top}
\begin{document}
\chapter{Introduction}


In today's world, the threat of a cyber attack can't be ignored. There are a plethora of companies that try to protect their customers from potential damage, such as getting infected by a virus, getting ransomware or malware or protect their business intelligence.

This thesis is concerned with another part of the cyber crime called phishing. Phishing is a social engineering attack to obtain sensitive information such as usernames, passwords, credit card details, bank account credentials for malicious reasons. Because of a large attack vector, there isn't a reliable way or a tool to prevent such an attack consistently in every sector.

This thesis will review several ways to detect a phishing attack, implement some of them and test them in the production environment with an association with a Trusted Network Solutions company. The final software could be easily used with another proxy or tool.

TODO: add why this work will be created and where will be it's place on market/who are possible customer/who will benefit from this.




% The word phishing is originated from the word fishing - attackers fishes on their victims. Phishers use a number of techniques to trick their possible victims: having a fake clone site like the original one with forms so a user will enter credentials to their site, or sending e-mails with messages to send money to some address.

% The total number of phish detected in 2Q 2018 was 233,040, compared to 263,538 in 1Q 2018. These totals exceed the 180,577 observed in 4Q 2017 and the 190,942 seen in 3Q 2017 [1].

% TODO: cite the organization APWG
% Nezisková organizácia APWG, ktorá ma za cieľ zjednotiť obranu voči cyberútokom, pravidelne vytvára štatistiky o aktuálnych trendoch a počtoch phishing útokov. Na číslach za uplynulý pol rok je vidno, že sa celkový počet pohybuje v podobnej rovine:

% Počet jednotlivých stránok má rôzne výkyvy, zatiaľ čo počet e-mailov sa postupne zvyšuje. Na základe analýz životnosti jednotlivých phishingových kampaní sa zistilo, že priemerná živostnosť je okolo 12 hodín [2]. Táto skutočnosť výrazne ovplyvňuje schopnosť automaticky detekovať phishingový útok (doplniť rozumný dôvod, citáciu).

% Rozdelenie ako sa môže phishingový útok detekovať, by sa dalo nasledovne kategorizovať:
% \begin{enumerate}
%     \item detekcia na základe blacklistov
%     \item detekcia na základe heuristiky
%     \item detekcia na základe vizuálnej podobnosti
%     \item detekcia na základe histórie a vlastnosti stránky
% \end{enumerate}

% Detekcia na základe blacklistov je najjednoduhším mechanizmom - zašleme request na API z niektorých poskytovateľov, poprípade sa pozrieme do lokálnej databázy. Nevýhoda týchto databáz je ich neaktuálnosť, pretože aby sa adresa objavila v zozname, tak musí prejsť procesom schválenia. Takisto tieto zoznamy neposkytujú ocharnu proti novým útokom a keď že trvácnosť stránok sa pohybuje v rádoch hodín, nie je toto dostatočná ochrana a mala by byť kombinovaná s niektorou inou.

% Detekcií na základe heuristiky môže byť veľké množstvo, v tejto práci sa budeme venovať najmä na zadávanie pravidiel na to, ako vyzerá phishingová URL a priraďovaniu hodnôt pre dané pravidlá. Jedná sa napríklad o:
% \begin{enumerate}
%     \item dĺžka URL
%     \item špeciálne znaky
%     \item hĺbka zanorenia
%     \item použitie IP adresy namiesto hostname
% \end{enumerate}


% TODO: spomenút výskum čo vyšiel v oktobér 2018 ohľadom toho že 35 percent phishingových stránok ma https a že tento trend ma stúpajúci charakter.

% TODO: spomenúť význam tejto práce, že sa ešte nenachádza komplexné riešenie pre integráciu do proxy pre lokálny trh

% TODO: poriadne odcitovať všetky horné poznatky aby to nebol kompilát.

% TODO: Pridať odstavec o detekcii na základe vizuálnej podobnosti.

% TODO: pridať odstavec ohľadom toho, že táto práca bude slúžiť ako ochrana pre užívateľov za proxy.

\chapter{Related work}

TODO: This chapter will be an exhaustive summary of all works and books that I could found and have read. It will be just a state and introduction to phishing problem, it won't be about what I will choose or what I will do later in a work, it will be only what have I done in a research.
\section{Phishing related books}
TODO: differs books vs applications/solutions so we can see if some similar app is existing
\section{Phishing related applications}
123

\chapter{Theoretical background}
TODO: this chapter will be about specific approaches and test - why I chose them and why are they beneficial for our app

\section{Phishing detection methods}
\subsection{List based approach}
\subsection{URL analysis}
\subsection{Analysis of a content}
\subsection{Visual analysis}
\subsection{Source-code analysis}
\subsection{Behavioral analysis}
\subsectoin{I will see later if something is missing}

\section{?Summary what we have chosen from 3.1?}

TODO: choose and discuss what we will use from these lists. Why we are using them. 

\chapter{Implementation}

\section{Used technologies}
TODO: discuss used technologies and their pros+cons

\section{Development environment}

TODO: add how to setup development environment and generally everything into appendix. If somebody in the future will try to extend this paper, it will be really helpfull (I think so)

\section{Modules for detection}
TODO: mention what repositories I've created and hopefully mention that somebody else uses it (safebrowsing-api, url parser...)

\subsection{Google safe browsing}
TODO: write about repository https://github.com/astaruch/safebrowsing-cpp
\subsection{Phishtank}

\subsection{URL anomalies detection}
\subsectoin{Behaviorial}
TODO: whois, dns resolving, IP, /robots.txt...

\subsection{And so on, this will be updated on the fly}

\section{?Combining it together to a backend/proxy/module?}
TODO: combine it and create configuration a production ready application from the modules. ?Probably use some machine learning to teach modules, what are best values for their detection approach, so I dont have to test it manually.? Write about https://github.com/astaruch/master-thesis-everything/tree/master/code/phishing-app . How to setup app.

\section{?Frontend?}
TODO: MAYBE IF THERE WILL BE TIME. Do a simple frontend for network administrator, so he can manually change behaviour of the app; e.g. add more values to some tests - enter google api key, refresh phishtank db, add new DNSBL and so on.

\chapter{Testing}

\section{Deployment}
TODO: ideally create a Docker image for simple deployment on something like phishing-proxy.staruch.sk

\section{Integration into proxy}
TODO: research possibility how to insert app into Kernun UTM/FW 

\section{Testing on real traffic}
TODO: ideally, at least 2 weeks of testing - around march/april

\section{Results}
TODO: prepare some graphs if this work is helpful or not, if speed of network is slower or unaffected, if customers are generally more happy, if newtork administrators can see some differences

\chapter{Summary}
TODO: discuss what I did and if this thesis fulfills its goal

% \chapter{These are}
% \section{the available}
% \subsection{sectioning}
% \subsubsection{commands.}
% \paragraph{Paragraphs and}
% \subparagraph{subparagraphs are available as well.}
% Inside the text, you can also use unnumbered lists,


% \chapter{Using lightweight markup}
% \shorthandoff{-}
% \begin{markdown*}{%
%   hybrid,
%   definitionLists,
%   footnotes,
%   inlineFootnotes,
%   hashEnumerators,
%   fencedCode,
%   citations,
%   citationNbsps,
% }

% If you decide that \LaTeX{} is too wordy for some parts of your
% document, there are [packages](https://www.ctan.org/pkg/markdown
% "Markdown") that allow you to use more lightweight markup next
% to it.

%  ![logo](fithesis/logo/mu/fithesis-base.pdf "The logo of the
%   Masaryk University")

% This is a bullet list. Unlike numbered lists, bulleted lists
% contain an **unordered** set of bullet points. When a bullet point
% contains multiple paragraphs, the list is typeset as follows:

%   * The first item of a bullet list

%     that spans several paragraphs,
%   * the second item of a bullet list,
%   * the third item of a bullet list.

% When none of the bullet points contains multiple paragraphs, the
% list has a more compact form:

%   * The first item of a bullet list,
%   * the second item of a bullet list,
%   * the third item of a bullet list.

% Unlike a bulleted list, a numbered list implies chronology or
% ordering of the bullet points. When a bullet point
% contains multiple paragraphs, the list is typeset as follows:

%   1. The first item of an ordered list

%      that spans several paragraphs,
%   2. the second item of an ordered list,
%   3. the third item of an ordered list.
%   #. If you are feeling lazy,
%   #. you can use hash enumerators as well.

% When none of the bullet points contains multiple paragraphs, the
% list has a more compact form:

%   6. The first item of an ordered list,
%   7. the second item of an ordered list,
%   8. the third item of an ordered list.

% Definition lists are used to provide definitions of terms. When
% a definition contains multiple paragraphs, the list is typeset
% as follows:

% Term 1

% :   Definition 1

% *Term 2*

% :   Definition 2

%         Some code, part of Definition 2

%     Third paragraph of Definition 2.

% When none of the bullet points contains multiple paragraphs, the
% list has a more compact form:

% Term 1
% :   Definition 1
% *Term 2*
% :   Definition 2

% Block quotations are used to include an excerpt from an external
% document in way that visually clearly separates the excerpt from
% the rest of the work:

% > This is the first level of quoting.
% >
% > > This is nested blockquote.
% >
% > Back to the first level.

% Footnotes are used to include additional information to the
% document that are not necessary for the understanding of the main
% text. Here is a footnote reference^[Here is the footnote.] and
% another.[^longnote]

% [^longnote]: Here's one with multiple blocks.

%     Subsequent paragraphs are indented to show that they
% belong to the previous footnote.

%         Some code

%     The whole paragraph can be indented, or just the first
%     line.  In this way, multi-paragraph footnotes work like
%     multi-paragraph list items.

% Citations are used to provide bibliographical references to other
% documents. This is a regular citation~[@borgman03, p. 123]. This is
% an in-text citation: @borgman03\. You can also cite several authors
% at once using both regular~[see @borgman03, p. 123; @greenberg98,
% sec.  3.2; and @thanh01] and in-text citations: @borgman03 [p.123;
% @greenberg98, sec. 3.2; @thanh01].

% Code blocks are used to include source code listings into the
% document:

%     #include <stdio.h>
%     #include <unistd.h>
%     #include <sys/types.h>
%     #include <sys/wait.h>
%     // This is a comment
%     int main(int argc, char **argv)
%     {
%         while (--c > 1 && !fork());
%         sleep(c = atoi(v[c]));
%         printf("%d\n", c);
%         wait(0);
%         return 0;
%     }

% There is an alternative syntax for code blocks that allows you to
% specify additional information, such as the language of the source
% code. This information can be used for syntax highlighting:

% ``` sh
% #!/bin/sh
% fac() {
%   if [ "$1" -leq 1 ]; then
%     echo 1
%   else
%     echo $(("$1" * fac $(("$1" - 1))))
%   fi
% }
% ``````````````

% ~~~~~~ Ruby
% # Here's a way to empty an array.
% joe = [ 'eggs.', 'some', 'break', 'to', 'Have' ]
% print(joe.pop, " ") while joe.size > 0
% print "\n"
% ~~~~~~

% \end{markdown*}
\shorthandon{-}



  \printbibliography[heading=bibintoc] %% Print the bibliography.


\end{document}
